\documentclass[a4paper,11pt]{article}

% ---------- Packages ----------
\usepackage[utf8]{inputenc}         % Encoding
\usepackage[T1]{fontenc}            % Font encoding
\usepackage[english]{babel}         % Language
\usepackage{amsmath, amssymb, amsthm} % Math symbols and theorems
\usepackage{mathtools}              % Extra math tools
\usepackage{physics}                % Derivatives, bras/kets, etc.
\usepackage{bm}                     % Bold math symbols
\usepackage{bbm}                    % Blackboard math symbols
\usepackage{hyperref}               % Clickable references
\usepackage{cleveref}               % Smart cross-referencing
\usepackage{geometry}               % Page layout
\usepackage{graphicx}               % For including figures
\usepackage{cite}                   % Better citation handling
\usepackage{enumitem}               % Custom lists
\usepackage{algorithm}
\usepackage{algpseudocode}
% ---------- Page Setup ----------
\geometry{margin=1in}

% ---------- Theorem Environments ----------
\theoremstyle{plain}
\newtheorem{theorem}{Theorem}[section]
\newtheorem{lemma}[theorem]{Lemma}
\newtheorem{proposition}[theorem]{Proposition}
\newtheorem{corollary}[theorem]{Corollary}

\theoremstyle{definition}
\newtheorem{definition}[theorem]{Definition}
\newtheorem{example}[theorem]{Example}

\theoremstyle{remark}
\newtheorem{remark}[theorem]{Remark}

% ---------- Common Math Shortcuts ----------
\newcommand{\R}{\mathbb{R}}
\newcommand{\N}{\mathbb{N}}
\newcommand{\Z}{\mathbb{Z}}
\newcommand{\Q}{\mathbb{Q}}
\newcommand{\C}{\mathbb{C}}
\newcommand{\E}{\mathbb{E}}
\newcommand{\Prob}{\mathbb{P}}
\newcommand{\eps}{\varepsilon}

\renewcommand{\vec}[1]{\bm{#1}}

% ---------- Document ----------
\title{Convolutional Wasserstein Notes}
\date{\today}

\begin{document}

\maketitle
These notes primarily follow from \cite[Chap. 4]{nutz2021introduction}. Here we focus on the case where $c(x, y) = \|x- y\|^2$ and our underlying space is $\R^d$.  Let $\mu_0, \mu_1 \in \mathcal{P}_2(\R^d)$ be two probability measures. Assume they are sufficiently regular to have densities and, with an abuse of notation, denote their densities with the same symbols. Denote $\Pi$ as the set of couplings between such measures. Let $\epsilon > 0$. We begin with the Entropic Optimal Transport problem (EOT): 
\begin{align}
\tag*{(EOT)} 
\min_{\pi \in \Pi} 
\left\{
    \int_{\R^d \times \R^d} \|x - y\|^2 \, d\pi 
    + \epsilon \, \mathrm{KL}(\pi, \mu_0 \otimes \mu_1)
\right\}.
\end{align}
By introducing an auxiliary reference measure $\mathcal{K}_\epsilon \in \mathcal{P}(\R^d \times \R^d)$ via $$d\mathcal{K}_\epsilon(x, y) = K_\epsilon(x, y) \, d(\mu_0 \otimes \mu_1),$$ where $K_\epsilon(x, y) = e^{-\frac{\|x-y\|^2}{\epsilon}}$, we can write (EOT) equivalently as 
\begin{align}
    \min_{\pi \in \Pi} \text{KL}(\pi, \mathcal{K}_\epsilon).
\end{align}
From \cite[Thm 4.2]{nutz2021introduction}, it follows that there exists a unique minimiser $\pi_\eps^\star$ to the above problem satisfying 
\begin{align*}
    \frac{d\pi_\epsilon^\star}{d(\mu_0 \otimes \mu_1)}(x,y) = e^{\frac{\varphi(x)}{\epsilon}} K_\epsilon(x, y) e^{\frac{\psi(y)}{\epsilon}},
\end{align*}
where $\varphi, \psi: \R^d \rightarrow \R$ are the EOT potentials, which are unique (up to an additive constant). Thus if we can determine the EOT potentials, we can solve the EOT problem.

Let $\mathcal{H}_t(x,y)$ be the heat kernel then, by \cite[Eq 7]{solomon2015convolutional}, defining $t \triangleq \frac{\epsilon}{2}$ we have
\begin{equation}
    K_\epsilon(x,y) \approx \mathcal{H}_\frac{\epsilon}{2}(x,y)
\end{equation}

For notational convenience, define $v_0(x) = e^{\frac{\varphi(x)}{\epsilon}}$ and $v_1(x) = e^{\frac{\psi(y)}{\epsilon}}$. The marginal constraints on $\pi_\epsilon^\star$ imply the following constraints: 
\begin{align*}
    & v_0(x) = \frac{\mu_0(x)}{\int_{\R^d}\mathcal{H}_\frac{\epsilon}{2}(x, y) v_1(y) \, dy} \\& 
    v_1(y) = \frac{\mu_1(y)}{\int_{\R^d}\mathcal{H}_\frac{\epsilon}{2}(x, y) v_0(x) \, dx},
\end{align*}
which can be solved using Sinkhorns algorithm \eqref{alg:sinkhorn} where the integral is a heat equation with initial condition $v_1(y)$ or $v_0(x)$. This is iterated until $(\varphi, \psi)$ is approximately a fixed point.
\begin{algorithm}[H]
\caption{Sinkhorn Algorithm}\label{alg:sinkhorn}
\begin{algorithmic}[1]
\Require $\mu_0, \mu_1 \in \mathcal{P}(\R^d)$, \ $\epsilon > 0$, \ $\text{tol} > 0$, \ $\text{max\_iter} \in \mathbb{N}_{+}$
\State Initialize $v_0^{(0)}(x) \gets 1$, \ $v_1^{(0)}(x) \gets 1$
\State Set $\text{res} \gets 1$, \ $k \gets 0$
\While{$\text{res} > \text{tol}$ \textbf{and} $k < \text{max\_iter}$}
    \State $v_0^{(k+1)}(x) \gets 
        \dfrac{\mu_0(x)}
        {\displaystyle\int_{\R^d} \mathcal{H}_{\frac{\epsilon}{2}}(x, y)\, v_1^{(k)}(y)\, dy}$
    \State $\text{res} \gets 
        \displaystyle\int_{\R^d} 
        \big|v_0^{(k+1)}(x) - v_0^{(k)}(x)\big|^2 \, dx$
    \State $v_1^{(k+1)}(y) \gets 
        \dfrac{\mu_1(y)}
        {\displaystyle\int_{\R^d} \mathcal{H}_{\frac{\epsilon}{2}}(x, y)\, v_0^{(k+1)}(x)\, dx}$
    \State $k \gets k + 1$
\EndWhile
\State Set $\varphi(x) \gets \epsilon \ln v_0^{(k)}(x)$, \quad $\psi(y) \gets \epsilon \ln v_1^{(k)}(y)$
\State \Return $(\varphi, \psi)$
\end{algorithmic}
\end{algorithm}

\bibliographystyle{plain} %
\bibliography{references}

\end{document}
